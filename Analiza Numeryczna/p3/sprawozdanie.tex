\documentclass[11pt]{article}
\usepackage[T1]{fontenc}
\usepackage[utf8]{inputenc}
\usepackage[polish]{babel}
\usepackage{amsthm}
\usepackage{amsmath}
\usepackage{program}
\usepackage{amsfonts}
\usepackage{titlesec}
\usepackage{graphicx}
\title{%
  Sprawozdanie \\
  \large P3.20 z analizy numerycznej\\
    Obliczanie $A^{-1}$ za pomocą rozkładu QR macierzy}
\author{Artur Derechowski\\Mateusz Markiewicz}
\begin{document}
\maketitle
\tableofcontents

\newtheorem{definition}{Definicja}[section]

\titleformat{\paragraph}
{\normalfont\normalsize\bfseries}{\theparagraph}{1em}{}
\titlespacing*{\paragraph}
{0pt}{3.25ex plus 1ex minus .2ex}{1.5ex plus .2ex}

%~~~~~~~~~~~~~~~~~~~~~~~~~~~~~~~~~~~~~~~~
\newpage
\section{Wstęp}
Macierz odwrotną do $A$, czyli taką $A^{-1}$, że $AA^{-1} = A^{-1}A = I$ można otrzymać na różne sposoby.
W naszej pracowni przedstawiamy dwa z nich, czyli rozkład QR i rozkład LU oraz porównanie tych dwóch metod.

\subsection{Cel zadania}
Zadanie zostało podzielone na dwie części. Wyznaczamy rozkład macierzy $A = QR$, gdzie $Q$ jest
macierzą ortogonalną, a $R$ macierzą górnotrójkątną.\\
Następnie wyznaczamy również rozkład macierzy $A = LU$, gdzie $L$ i $U$ są odpowiednio macierzami
dolno i górnotrójkątnymi.\\
Następnie wykorzystujemy oba te rozkłady do policzenia macierzy odwrotnej do $A$ i badamy je pod względem
dokładności.

\subsection{Streszczenie sprawozdania}

\section{Opis teoretyczny problemu}
\subsection{Wprowadzenie do zagadnienia}

\subsection{?}

\section{Rozkład QR}
\subsection{Wstęp}
Daną macierz $A$ można jednoznacznie rozłożyć na iloczyn macierzy $Q$ i $R$, takich, że $A=QR$.
Dodatkowo, macierz $Q$ jest ortogonalna, czyli jest odwrotna do swojej transpozycji, spełnia równanie $QQ^T=I$.
Macierz $R$ jest natomiast górnotrójkątna. \\
Dzięki temu można łatwo rozwiązać układ równań $Ax=b$:
\begin{align*}
Ax=b \\
QRx=b \\
Rx = Q^Tb
\end{align*}
Gdy układ równań jest trójkątny, kolejne zmienne można wyznaczyć podstawieniami w sumarycznym czasie $O(n^2)$.\\

Podobnie, mając macierz $A=QR$, można wyliczyć odwrotność macierzy $A$, co jest naszym zadaniem.
\begin{align*}
AA^{-1}=I \\
QRA^{-1}=I \\
A^{-1}=R^{-1}Q^T
\end{align*}

\subsection{Przekształcenia Householdera}
Aby uzyskać macierz górnotrójkątną $R$, w każdym kroku algorytmu będziemy zerować
dolną część jednej kolumny macierzy $A$. Używamy do tego przekształceń Householdera,
które jest odbiciem, czyli przekształceniem ortogonalnym. 
Przykładowo, po zastosowaniu jednego przekształcenia Householdera,
macierz $H_1A$ będzie wyglądała następująco: 
\begin{center}
\begin{math}
\begin{bmatrix}
    a_{11} & a_{12} & \dots  & a_{1n} \\
    0 & a_{22} & \dots  & a_{2n} \\
    \vdots & \vdots & \ddots & \vdots \\
    0 & a_{n2} & \dots  & a_{nn}
\end{bmatrix}
\end{math}
\end{center}
Aby uzyskać takie przekształcenie mnożymy macierz $A$ przez macierz Householdera postaci 
\begin{center}
$H=I-2v*v^T$
\end{center}
gdzie $v$ jest wektorem jednostkowym.\\

\subsection{Przebieg algorytmu}
Do rozkładu QR stosuje się przekształcenia Householdera na wektorze 
\begin{equation*}
u=x-\|x\|e_1
\end{equation*}
gdzie $x$ jest pierwszą kolumną macierzy $A$, a $e_1 = [1, 0, ..., 0]^T$.\\

Gdy normalizujemy wektor $u$, macierz przekształcenia $H$ dana jest wzorem:
\begin{equation*}
H=I-2\frac{u}{\|u\|}\frac{u^T}{\|u^T\|} = I-2\frac{uu^T}{u^Tu}
\end{equation*}
Można pokazać, że po tym przekształceniu macierz $H_1A$ będzie miała wyzerowaną
pierwszą dolną kolumnę poza najwyższym polem, czyli będzie odpowiedniej postaci
do dalszego przekształcania na macierz górnotrójkątną. \\

W kolejnych krokach algorytmu 
stosujemy przekształcenia Householdera na macierzy $A$ bez lewej kolumny i górnego wiersza.
Wtedy uzyskana macierz będzie jednak mniejsza od macierzy $A$, więc aby nie zmieniać ''lewych'' kolumn,
które zostały wcześniej dobrze dopasowane, dopełniamy macierz $H_i$ macierzą identycznościową.
Wtedy macierz, przez którą w każdym kroku mnożymy dotychczas uzyskaną macierz wygląda następująco:
\begin{center}
\begin{math}
H_k :=
\begin{bmatrix}
    I_{n-k} & 0 \\
    0 & H_k \\
\end{bmatrix}
\end{math}
\end{center}

Stosując $n-1$ kolejnych przekształceń Householdera $H_i$ otrzymujemy wynikowo macierz górnotrójkątną $R$, 
a także ortogonalną macierz $Q$, czyli rozkład, którego szukamy w następujący sposób:
\begin{center}
$R = H_{n-1}...H_2H_1A $ \\
$Q = H_1H_2...H_{n-1}$
\end{center}

Wynikowe macierze w rozkładzie $QR$ można więc przedstawić jako iloczyn macierzy Householdera.

Można zobaczyć, że mając rozkład macierzy na iloczyn $QR$ spełniający wyżej wymienione własności,
macierz odwrotną $A^{-1}$ można wyliczyć jako $A^{-1}=R^{-1}Q^T$. Następnym krokiem algorytmu jest więc
odwrócenie macierzy górnotrójkątnej $R$. Można to zrobić w następujący sposób:\\
\begin{program}
  \FOR i:=1 \TO n \DO
	A_{ii} = 1/A_{ii}
  \FOR i:=n-1 \STEP -1 \TO 1 \DO
	\FOR j:=n \STEP -1 \TO i+1 \DO
	  s:=0
	  \FOR k:=i+1 \TO j \DO
	    s:= s+A_{ik}*A_{kj}
	  A_{ij} = - A_{ii}*s
\end{program}

Ten algorytm, zastosowany do macierzy górnotrójkątnej, wylicza jej odwrotność w czasie $O(n^3)$.\\

Cały algorytm wyznaczania macierzy odwrotnej za pomocą rozkładu $QR$ tworzy $n$ macierzy $H_i$, 
które można wyliczyć w czasie $O(n^2)$, następnie wykonuje operacje w czasie $O(n^3)$,
więc cała złożoność algorytmu to $O(n^3)$.

\subsection{Gram-Schmidt}
Znaną metodą ortogonalizacji macierzy jest proces Grama-Schmidta i również w ten sposób można uzyskać rozkład $QR$.
Nie jest to jednak zalecane, ponieważ w wyniku tych przekształceń mogą powstawać bardziej znaczące błędy numeryczne.
Można to wywnioskować interpretując proces Grama-Schmidta jako odejmowanie od wektora jego rzutów na poprzednio
uzyskane wektory. Wtedy, gdy dwa wektory były ''prawie'' liniowo zależne, odejmujemy niemal cały wektor, co powoduje
brak stabilności numerycznej.\\ 

\section{Rozkład LU}
\subsection{Wstęp}
Rozkład $LU$ macierzy $A$ polega na znalezieniu macierzy dolnotrójkątnej L oraz górnotrójkątnej U, takich że ich iloczyn będzie macierzą A, czyli:
\begin{center}
\begin{math}
\begin{bmatrix}
    a_{11} & a_{12} & \dots  & a_{1n} \\
    a_{21} & a_{22} & \dots  & a_{2n} \\
    \vdots & \vdots & \ddots & \vdots \\
    a_{n1} & a_{n2} & \dots  & a_{nn}
\end{bmatrix}
=
\begin{bmatrix}
    1 & 0  & \dots  & 0 \\
    l_{21} & 1 & \dots  & 0 \\
    \vdots & \vdots & \ddots & \vdots \\
    l_{n1} & l_{n2} & \dots  & 1
\end{bmatrix}
\times
\begin{bmatrix}
    u_{11} & u_{12} & \dots  & u_{1n} \\
    0 & u_{22} & \dots  & u_{2n} \\
    \vdots & \vdots & \ddots & \vdots \\
    0 & 0 & \dots  & u_{nn}
\end{bmatrix}
\end{math}
\end{center}

\subsection{Metoda Doolittle’a}
Metoda Doolittle’a polega na naprzemiennym wyznaczaniu kolejnych wierszy macierzy U oraz kolumn macierzy L. Szczegółowy algorytm wygląda następująco:
\begin{program}
  \FOR i:=1 \TO n \DO
	u_{ii} = a_{ii} - \sum_{k=1}^{i-1} l_{ik} u_{ki}\\
	l_{ii} = 1\\
	\FOR j:=i+1 \TO n \DO
		u_{ij} = a_{ij} - \sum_{k=1}^{i-1} l_{ik} u_{kj}\\
		l_{ji} = \frac{1}{u_{ii}} (a_{ji} - \sum_{k=1}^{i-1} l_{jk} u_{ki})\\
\end{program}
Na podstawie algorytmu widać, że rozkład macierzy rozmiaru $n$ wymaga $O(n^{3})$ operacji. Duża liczba operacji wpływa negatywnie zarówno na czas działania, jak również dokładność obliczeń. 

\subsection{Obliczenie macierzy odwrotnej za pomocą rozkładu LU}
Dla danej macierzy kwadratowej $A$ rozmiaru $n$ macierz odwrotna $A^{-1}$ to macierz kwadratowa tego samego rozmiaru spełniająca równość
\begin{center}
$A \times A^{-1} = Id = A^{-1} \times A$
\end{center}
Powyższą równość możemy zapisać w postaci:
\begin{center}
\begin{math}
A \times 
\begin{bmatrix}
    x_{11} & x_{12} & \dots  & x_{1n} \\
    x_{21} & x_{22} & \dots  & x_{2n} \\
    \vdots & \vdots & \ddots & \vdots \\
    x_{n1} & x_{n2} & \dots  & x_{nn}
\end{bmatrix}
=
\begin{bmatrix}
    1 & 0 & \dots  & 0 \\
    0 & 1 & \dots  & 0 \\
    \vdots & \vdots & \ddots & \vdots \\
    0 & 0 & \dots  & 1
\end{bmatrix}
\end{math}
\end{center}
stąd otrzymujemy $n$ układów równań:
\begin{center}
\begin{math}
A \times
\begin{bmatrix}
    x_{11} \\
    x_{21} \\
    \vdots \\
    x_{n1} 
\end{bmatrix}
=
\begin{bmatrix}
    1 \\
    0 \\
    \vdots \\
    0
\end{bmatrix}
\hspace{1em}
A \times
\begin{bmatrix}
    x_{12} \\
    x_{22} \\
    \vdots \\
    x_{n2} 
\end{bmatrix}
=
\begin{bmatrix}
    0 \\
    1 \\
    \vdots \\
    0
\end{bmatrix}
\hspace{0.7em} \hdots \hspace{0.8em}
A \times
\begin{bmatrix}
    x_{1n} \\
    x_{2n} \\
    \vdots \\
    x_{nn} 
\end{bmatrix}
=
\begin{bmatrix}
    0 \\
    0 \\
    \vdots \\
    1
\end{bmatrix}
\end{math}
\end{center}

\paragraph{Rozwiązywanie układów równań za pomocą rozkładu LU}

Korzystając z własności macierzy wiemy, że:
\begin{center}
$A \times X = B$ \\
$(L \times U) \times X = B$ \\
$L \times (U \times X) = B$
\end{center}
Stąd znajdując wektor $Y$ taki, że $L \times Y = B$ możemy wyznaczyć wektor X z własności $U \times X = Y$. Wektory $X$ oraz $Y$ łatwo wyznaczyć z zależności:
\begin{center}
$y_{1} = b_{1}$\\
\vspace{0.3em} $y_{i} = b_{i} - \sum_{j=1}^{i-1} l_{ij} y_{j}$, dla $i = 2,3,...,n$\\
\vspace{0.3em} $x_{n} = \frac{y_{n}}{u_{nn}}$\\
\vspace{0.3em} $x_{i} = \frac{1}{u_{ii}} \Big( y_{i} - \sum_{j=i+1}^{n} u_{ij} x_{j} \Big)$ dla $i = n-1,n-2,...,1$
\end{center}

\paragraph{Wyznaczanie $A^{-1}$}
Rozwiązując powyższe $n$ układów równań otrzymamy $n$ wektorów $X_{1}, X_{2},..., X_{n}$. Niech $X = [ X_{1} \vert X_{2} \vert ... \vert X_{n} ]$, wtedy $X = A^{-1}$, czyli wyznaczyliśmy macierz odwrotną do macierzy $A$. Dla macierzy $A$ rozmiaru $n \times n$ rozkład na macierze $L, U$ wymaga $O(n^{3})$ operacji. Obliczenie $A^{-1}$ wymaga rozwiązania $n$ układów równań, a rozwiązanie każdego z tych układów wymaga $O(n^{2})$ operacji, stąd cały algorytm obliczania $A^{-1}$ za pomocą rozkładu LU jest rzędu $O(n^{3})$.
\subsection{?}

\section{Podsumowanie}

\section{Literatura}

1) D. Kincaid, W. Cheney, Analiza numeryczna, WNT, 2005.\\
2) A. Schegel (aaronsc32), QR Decomposition with Householder Reflections, RPubs, 2018.\\
3) Mathematics Source Library C\&ASM, mymathlib.com, 2004.\\
4) D. Bindel, Matrix Computations (CS6210), Cornell University, Sep 28 2012.\\ 


	 

\end{document}
