%%%%%%%%%%%%%%%%%%%%%%%%%%%%%%%%%%%%%%%%%
% Short Sectioned Assignment
% LaTeX Template
% Version 1.0 (5/5/12)
%
% This template has been downloaded from:
% http://www.LaTeXTemplates.com
%
% Original author:
% Frits Wenneker (http://www.howtotex.com)
%
% License:
% CC BY-NC-SA 3.0 (http://creativecommons.org/licenses/by-nc-sa/3.0/)
%
%%%%%%%%%%%%%%%%%%%%%%%%%%%%%%%%%%%%%%%%%

%----------------------------------------------------------------------------------------
%	PACKAGES AND OTHER DOCUMENT CONFIGURATIONS
%----------------------------------------------------------------------------------------

\documentclass[paper=a4, fontsize=11pt]{scrartcl} % A4 paper and 11pt font size
\usepackage[utf8]{inputenc}
\usepackage[MeX]{polski}
\usepackage[T1]{fontenc} % Use 8-bit encoding that has 256 glyphs
\usepackage{fourier} % Use the Adobe Utopia font for the document - comment this line to return to the LaTeX default
 % English language/hyphenation
\usepackage{amsmath,amsfonts,amsthm} % Math packages
\usepackage{graphicx} %images
\usepackage{placeins}%for direct positioning
\usepackage{lipsum} % Used for inserting dummy 'Lorem ipsum' text into the template

\usepackage{sectsty} % Allows customizing section commands
\allsectionsfont{\centering \normalfont\scshape} % Make all sections centered, the default font and small caps

\usepackage{fancyhdr} % Custom headers and footers
\pagestyle{fancyplain} % Makes all pages in the document conform to the custom headers and footers
\fancyhead{} % No page header - if you want one, create it in the same way as the footers below
\fancyfoot[L]{} % Empty left footer
\fancyfoot[C]{} % Empty center footer
\fancyfoot[R]{\thepage} % Page numbering for right footer
\renewcommand{\headrulewidth}{0pt} % Remove header underlines
\renewcommand{\footrulewidth}{0pt} % Remove footer underlines
\setlength{\headheight}{13.6pt} % Customize the height of the header

\numberwithin{equation}{section} % Number equations within sections (i.e. 1.1, 1.2, 2.1, 2.2 instead of 1, 2, 3, 4)
\numberwithin{figure}{section} % Number figures within sections (i.e. 1.1, 1.2, 2.1, 2.2 instead of 1, 2, 3, 4)
\numberwithin{table}{section} % Number tables within sections (i.e. 1.1, 1.2, 2.1, 2.2 instead of 1, 2, 3, 4)

\setlength\parindent{0pt} % Removes all indentation from paragraphs - comment this line for an assignment with lots of text

%----------------------------------------------------------------------------------------
%	TITLE SECTION
%----------------------------------------------------------------------------------------

\newcommand{\horrule}[1]{\rule{\linewidth}{#1}} % Create horizontal rule command with 1 argument of height

\title{	
\normalfont \normalsize 
\textsc{Uniwersytet Wrocławski} \\ [25pt] % Your university, school and/or department name(s)
\horrule{0.5pt} \\[0.4cm] % Thin top horizontal rule
\huge Interpolacja Lagrange'a i Newtona \\
\large Pracownia 2.7 \\ % The assignment title
\horrule{2pt} \\[0.5cm] % Thick bottom horizontal rule
}

\author{Artur Derechowski} % Your name

\date{\normalsize\today} % Today's date or a custom date

\begin{document}

\maketitle % Print the title

%----------------------------------------------------------------------------------------
%	PROBLEM 1
%----------------------------------------------------------------------------------------

\section{Treść}
Zrealizować algorytmy obliczania wartości wielomianu podanego za pomocą wzoru interpolacyjnego
Lagrange’a oraz zamiany postaci Lagrange’a na postać Newtona. Porównać dokładności wyników uzyskanych
za pomocą obu wzorów m.in. dla funkcji ${f_1(x) = (1+25x^2)^{-1}}$
i ${f_2(x) = arctg(x)}$.

\subsection{Postać Lagrange'a}
Wielomian interpolujący ${n+1}$ punktów ${(x_0, y_0), (x_1, y_1), ... , (x_n, y_n)}$ można zapisać
w postaci interpolacyjnej Lagrange'a, danej wzorem:
\begin{align} 
 p(x)= \sum_{i=0}^{n} f_i L_i(x), &&	\label{lag}
 L_i(x) = \prod_{\substack{j=0 \\ j \neq i}}^n \frac{x-x_j}{x_i-x_j}
\end{align}

Można sprawdzić, że ten wielomian przyjmuje wartości $y_i$ w punktach $x_i$ oraz, że jest
stopnia $\leqslant n$
Jest to jedyny wielomian stopnia ${\leqslant n}$ interpolujący zadane punkty. \medbreak
 
\subsection{Postać Newtona}
Ten sam wielomian może zostać zapisany w postaci Newtona jako:
\begin{align} 
 p(x)= \sum_{i=0}^{n} a_i \prod_{j=0}^{i-1} (x-x_j), && \label{new}
 a_k = \sum_{i=0}^k \frac{f(x_i)} {\prod_{\substack{j=0, j \neq i}}^k (x_i - x_j)}
\end{align}
gdzie współczynniki ${a_i}$ są ilorazami różnicowymi, które można również definiować rekurencyjnie jako:

\begin{align}
a_i = f[x_0,..., x_i] &&
f[x_0,..., x_i] = \frac{f[x_0,..., x_{i-1}] - f[x_1,..., x_i]}{x_0 - x_i}
\end{align}

\section{Zamiana postaci Lagrange'a na postać Newtona}
Postać Lagrange'a podana we wzorze \ref{lag} może być również zapisana jako:
\begin{align} 
 p(x)= \sum_{i=0}^{n} \sigma_i \prod_{\substack{j=0 \\ j \neq i}}^n (x-x_j), &&	\label{lsig}
 \sigma_i = \frac{f(x_i)}{\prod_{\substack{j=0 \\ j \neq i}}^n (x_i-x_j)}
\end{align}

Mając zapisane $\sigma_i$ w tej postaci można poprzez proste przekształcenie otrzymać
również współczynniki w postaci Newtona.
\begin{align} 
 a_k = \sum_{i=0}^k \frac{f(x_i)} {\prod_{\substack{j=0, j \neq i}}^k (x_i - x_j)} \label{a}
 = \sum_{i=0}^k \sigma_i \prod_{j=k+1}^n (x_i-x_j)
\end{align}

Wtedy, z równań \ref{a} oraz \ref{new} postać Newtona wielomianu interpolacyjnego
dana jest wzorem:
\begin{align} 
 p(x)= \sum_{i=0}^{n} a_k \prod_{j=0}^{i-1} (x-x_j) \label{nsig}
 = \sum_{i=0}^k \frac{f(x_i)} {\prod_{\substack{j=0, j \neq i}}^k (x_i - x_j)} \prod_{j=0}^{i-1} (x-x_j)
\end{align}

W dalszej części pracy będą rozważane wzory na postać Newtona podane w równaniu
\ref{nsig} oraz na postać Lagrange'a w równaniu \ref{lsig}.

\section{Badane funkcje}

Wiadomo, że n-ty błąd interpolacji dany jest wzorem:
\begin{align} 
 f(x) - p(x) = \frac{f^{(n+1)}(\xi)}{(n+1)!} \prod_{i=0}^n (x-x_i)
\end{align}
zależy on więc od pochodnej n-tego stopnia funkcji $f$. 
Dobierzemy więc odpowiednie funkcje, dla których ta pochodna może być dowolnie duża
(przy rosnącym n), aby uwidocznić różnicę pomiędzy funkcją a wielomianem ją interpolującym.

\subsection{Runge}
Funkcja Rungego dana jest wzorem:
\begin{align}
 f(x) = \frac{1}{1+25x^2} 
\end{align}
Można pokazać, że jej kolejne pochodne są rozbieżne, na przykład 10-ta pochodna
w pobliżu zera osiąga wartość ponad $2*10^13$.

\subsection{arctan(x)}
Kolejne pochodne funkcji $arctan$ również gwałtownie rozbiegają w pobliżu zera,
dla przykładu 10-ta pochodna osiąga wartość ponad $2*10^5$. Można zauważyć, 
że zasada działania jest podobna do tej dla funkcji Rungego, ponieważ 
\begin{align}
 (arctan(x))' = \frac{1}{1+x^2}
\end{align}
czyli również ma $1+x^2$ w mianowniku i stały licznik, kolejne pochodne będą więc
miały ograniczony mianownik w okolicy zera, licznik natomiast rośnie.
Funkcja Rungego pokazuje jednak tę różnicę o wiele lepiej, dając o wiele mniej
dokładne wyniki interpolacji.

\subsection{sin(x)}
Dla porównania wzięta została również jedna funkcja, której pochodne nie rosną gwałtownie.
\begin{align}
 (sin(x))' = cos(x) &&
 (sin(x))'' = -sin(x) &&
\end{align}
kolejne pochodne zostaną więc ograniczone do $\pm1$ w okolicy zera.

\section{Pomiary}

\subsection{Lagrange a Newton}
Poniżej przedstawiono pomiary dla funkcji Rungego. Pomiary przebiegają
co 5 stopni wielomianu interpolacyjnego, na przedziale $(-1,1)$ liczony jest błąd
jako różnica w pomiędzy wielomianem interpolującym postaci a funkcją.
Przedstawione jest to odpowiednio dla postaci Newtona i Lagrange. 
Różnica wyrażona jest jako całka oznaczona na przedziale $(-1,1)$.
Wykres \ref{lagnew} został przedstawiony w skali logarytmicznej, aby pokazać od którego miejsca
widoczna jest różnica pomiędzy postaciami.
Widać, że postać Newtona daje dokładniejsze wyniki od postaci Lagrange'a.

\subsection{Porównanie funkcji}
Błąd dokładności zależy od stopnia wielomianu interpolującego, ale także od interpolowanej funkcji.
Poniższa tabela przedstawia różnicę pomiędzy wielomianem interpolującym n-tego stopnia, a funkcjami:
$sin(x)$, $arctan(x)$, $runge(x)$.

\section{Wnioski}

\begin{figure}[h!]
  \includegraphics[width=\linewidth]{lagnew.png}
  \caption{Porównanie dokładności postaci Newtona i Lagrange'a dla funkcji Rungego}
  \label{lagnew}
\end{figure} 


\begin{thebibliography}{9}

\bibitem{artykul}
  Wilhelm Werner
  \textit{Polynomial Interpolation: Lagrange versus Newton}, \\
  Mathematics of Computation, volume 43, number 167, July 1984
\end{thebibliography}

\end{document}
